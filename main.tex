\documentclass{article}

\usepackage{graphicx}
\usepackage{float}
\usepackage[utf8]{inputenc}
\usepackage[OT4]{fontenc}
\usepackage[polish]{babel}
\usepackage{cancel}
\usepackage{amsmath}
\usepackage[legalpaper, margin=2cm]{geometry}

\setlength{\parindent}{0pt}

\begin{document}
\section{Aproksymacja}
\subsection{Aproksymacja średniokwadratowa dla funkcji zadanej dyskretnie.}
Dla danej funkcji:
\begin{table}[H]
\begin{tabular}{|c|c|}
\hline
x  & f(x) \\ \hline
-1 & 0    \\ \hline
0  & -1   \\ \hline
1  & 0    \\ \hline
2  & 1    \\ \hline
\end{tabular}%
\end{table}
Znajdź jej aproksymacje metodą średniokwadratową dla postaci:\\
\subsubsection{$f(x)=ax+bx^2$}
$f(a,b) = ((y(-1)-0)^2 + (y(0)- -1)^2 + (y(1)-0)^2 + (y(2)-1)^2)*\frac{1}{4}$\\
$f(a,b) = ((-a+b)^2 + (1)^2 + (a+b)^2 + (2a+4b-1)^2)*\frac{1}{4}$\\
$f(a,b) = (a^2+b^2\cancel{-2ab} + 1 + a^2+b^2\cancel{+2ab} + 4a^2+16ab-4a+16b^2-8b+1)*\frac{1}{4}$\\
$f(a,b) = (\frac{3}{2}a^2+4ab-a+\frac{9}{2}b^2-2b+\frac{1}{2})$\\
$\frac{d}{da}f(a,b)=3a+4b-1$\\
$\frac{d}{db}f(a,b)=4a+9b-2$\\

\begin{equation*}\begin{cases}
    12a+16b-4=0 \\
    16a+36b-8=0
\end{cases}\end{equation*}

\begin{equation*}\begin{cases}
    a=\frac{1}{11}\\
    b=\frac{2}{11}
\end{cases}\end{equation*}\\


Uwaga: Teoretycznie trzeba robić kolejne kroki, ale łatwo pokazać, że większe b stworzy dużo mniej dokładne wyniki, a te wartości to jedyne rozwiązania. Dodatkowo nie trzeba dzielić całości przez 4, bo i tak w pochodnych to nie ma znaczenia, bo po 2giej stronie jest 0 i można sobie pomnożyć.

\subsubsection{$f(x)=a+bx^3$}
$f(a,b) = (y(-1)-0)^2 + (y(0)- -1)^2 + (y(1)-0)^2 + (y(2)-1)^2$\\
$f(a,b) = (a-b)^2 + (a- -1)^2 + (a+b)^2 + (a+8b-1)^2$\\
$f(a,b) = 4a^2+16ab+66b^2-16b+2$
$\frac{d}{da}f(a,b)=8a+16b$\\
$\frac{d}{db}f(a,b)=16a+132b-16$\\

\begin{equation*}\begin{cases}
    8a+16b=0 |*\frac{1}{2}\\
    16a+132b-16=0  |*\frac{1}{4}
\end{cases}\end{equation*}

\begin{equation*}\begin{cases}
    4a+8b=0\\
    4a+33b-4=0
\end{cases}\end{equation*}

\begin{equation*}\begin{cases}
    a=-\frac{8}{25}\\
    b=\frac{4}{25}
\end{cases}\end{equation*}

\subsubsection{$f(x) = ax+b$}
$f(a,b)=(-a+b-0)^2+(b-(-1))^2+(a+b-0)^2+(2a+b-1)^2$\\
$f(a,b)=6a^2+4ab-4a+4b^2+2$\\
$\frac{d}{da}f(a,b)=12a+4b-4$\\
$\frac{d}{db}f(a,b)=4a+8b$\\

\begin{equation*}\begin{cases}
    12a+4b-4=0\\
    4a+8b=0
\end{cases}\end{equation*}

\begin{equation*}\begin{cases}
    a=\frac{2}{5}\\
    b=-\frac{1}{5}
\end{cases}\end{equation*}

\subsection{Aproksymacja średniokwadratowa dla funkcji w przedziale}
Dla funkcji $f(x)=x^4, x\in<0; 1>$ Znajdź aproksymacją średniokwadratowa funkcji $f(x)=ax^2$. Znajdź, dla jakiej wartości x błąd aproksymacji będzie największy.\\
$f(a) = (\int_{0}^{1}(ax^2-x^4)dx)^2$\\
$f(a) = \frac{a^2}{9}-\frac{2a}{15}+\frac{1}{25}$\\
$\frac{d}{da}f(a) = \frac{2a}{9}-\frac{2}{15}$

\begin{equation*}\begin{cases}
    \frac{a}{3}-\frac{1}{5} = 0
\end{cases}\end{equation*}

\begin{equation*}\begin{cases}
    a = 0.6
\end{cases}\end{equation*}

Musimy znaleźć $max|(0.6x^2-x^4)|\in\{0; 1\}$, ale że nie chce mi się bawić w pochodną dla wartości bezwzględnej, znajdę minima i maksima funkcji i wezmę wartość bezwzględną z nich.
$f(x) = 0.6x^2$
$g(x) = 0.6x^2-x^4$
$\frac{d}{dx}g(x) = 1.2x-4x^3$

\begin{equation*}\begin{cases}
    1.2x-4x^3 = 0
\end{cases}\end{equation*}

\begin{equation*}\begin{cases}
    x \in \{-\sqrt{0.3}, 0, \sqrt{0.3}\}
\end{cases}\end{equation*}

$-\sqrt{0.3}<0$, więc jest poza przedziałem. Należy równierz dodać 0 i 1, jako początek i koniec przedziału.

\begin{equation*}\begin{cases}
    |f(0)| = 0\\
    |f(\sqrt{0.3})|=0.09\\
    |f(1)| = 0.4
\end{cases}\end{equation*}

Dla x=1 aproksymacja jest obarczona największym błędem.

\subsection{Jednostajna}
\textbf{UWAGA: Idiota programista nie zrobił, nie używać}
W aproksymacji jednostajnej należy stworzyć jak najmniejszy maksymalny błąd.\\
\subsubsection{Metoda szeregów potęgowych}
Nie rozumiem do końca, co chciał, bo szereg potęgowy to dosłownie wielomian.\\
$\sqrt{x}=ax^2+bx+c$\\
$f(x)=\sqrt{x}-(ax^2+bx+c), x\in<0;1>$\\
$\frac{d}{dx}=\frac{1}{2\sqrt{x}}-2ax-b$

\begin{equation*}\begin{cases}
    \frac{1}{2\sqrt{x}}-2ax-b=0
\end{cases}\end{equation*}


\section{Lagrange}
\subsection{Wyznacz współczynniki wielomianu}
Dla następujących danych:
\begin{table}[H]
\begin{tabular}{|c|c|}
\hline
x  & f(x) \\ \hline
-4 & 2    \\ \hline
-2 & -3   \\ \hline
3  & 2    \\ \hline
6  & -3   \\ \hline
\end{tabular}
\end{table}
Wyznacz współczynniki wielomianu Langrange'a\\
$l0(x)=\frac{x- (-2)}{-4- (-2)}*\frac{x-3}{-4-3}*\frac{x-6}{-4-6} = -\frac{1}{140}x^3+\frac{1}{20}x^2-\frac{9}{35}$\\ %-4
$l1(x)=\frac{x- (-4)}{-2- (-4)}*\frac{x-3}{-2-3}*\frac{x-6}{-2-6} = \frac{1}{80}x^3-\frac{1}{16}x^2-\frac{9}{40}x+\frac{9}{10}$\\ %(-2)
$l2(x)=\frac{x- (-4)}{3- (-4)}*\frac{x- (-2)}{3- (-2)}*\frac{x-6}{3-6} = -\frac{1}{105}x^3+\frac{4}{15}x+\frac{16}{35}$\\ %3
$l3(x)=\frac{x- (-4)}{6- (-4)}*\frac{x- (-2)}{6- (-2)}*\frac{x-3}{6-3} = \frac{1}{240}x^3+\frac{1}{80}x^2-\frac{1}{24}x-\frac{1}{10}$\\ %6
$L(x)=2*l0(x)+-3*l1(x)+2*l2(x)+-3*l3(x)$\\
$L(x)=-\frac{1}{12}x^3+\frac{1}{4}x^2+\frac{4}{3}x-2$\\
Uwaga: Można sprawdzić, czy wielomian jest poprawnie obliczony wstawiając do niego wartości z tabelki. Powinny wychodzić identyczne wartości f(x).
\subsection{Wyznacz pochodną w punkcie}
Dla danych:
\begin{table}[H]
\begin{tabular}{|c|c|}
\hline
x  & f(x) \\ \hline
-1 & -1   \\ \hline
1  & 0    \\ \hline
3  & -1   \\ \hline
4  & 1    \\ \hline
\end{tabular}
\end{table}
Wyznacz pochodną dla x=1 i porównaj wynik w tym punkcie na podstawie ilorazu różnicowego centralnego.\\
\subsubsection{Używając wielomianu}
$l0(X)=\frac{x-1}{-1-1}*\frac{x-3}{-1-3}*\frac{x-4}{-1-4} = \frac{1}{40}(-x^3+8x^2-19x+12)$\\
$l1(X)=\frac{x- (-1)}{1- (-1)}*\frac{x-3}{1-3}*\frac{x-4}{1-4} = \frac{1}{12}(x^3-6x^2+5x+12)$\\
$l2(X)=\frac{x- (-1)}{3- (-1)}*\frac{x-1}{3-1}*\frac{x-4}{3-4} = \frac{1}{8}(-x^3+4x^2+x-4)$\\
$l3(X)=\frac{x- (-1)}{4- (-1)}*\frac{x-1}{4-1}*\frac{x-3}{4-3} = \frac{1}{15}(x^3-3x^2-x+3)$\\
$L(x)=-1*l0(x)+\cancel{0*l1(x)}+-1*l2(x)+1*l3(x)$\\
$L(x)=\frac{13x^3}{60}-\frac{9x^2}{10}+\frac{17x}{60}+\frac{2}{5}$\\
$L'(x)=\frac{13x^2}{20}-\frac{9x}{5}+\frac{17}{60}$\\
$L'(1)=\frac{13}{20}-\frac{9}{5}+\frac{17}{60}=-\frac{13}{15}$
\subsubsection{Używając ilorazu różnicowego centralnego}
Uwaga: Sekcja zrobiona przy pomocy AI, ale pokrywa się mniej-więcej z książką\\
Znajdujemy takie punkty, żeby $|x-x_1|=|x-x_2|$, dla tego przypadku $x_1 = -1$ i $x_2 = 3$. Odległość od x $h=2$\\
$f'(1)=\frac{f(x+h)-f(x-h)}{2h}=\frac{f(3)-f(-1)}{2*2}=\frac{-1-(-1)}{4}=0$\\
Co jest różne w porównaniu do wartości wyliczonej z wielomianu. Jest to jednak do przewidzenia, ponieważ jeżeli wyświetlimy wykres wielomianu, to nie jest to maksimum lokalne, ale wg funkcji dyskretnej jest.

\section{Newton}
Dla danej funkcji:
\begin{table}[H]
\begin{tabular}{|c|c|}
\hline
x  & f(x) \\ \hline
-1 & 0    \\ \hline
0  & -1   \\ \hline
1  & 0    \\ \hline
2  & 1    \\ \hline
\end{tabular}%
\end{table}
$n0=0$\\
$n1=(\cancel{\frac{0}{-1-0}} + \frac{-1}{0- (-1)})(x-(-1))$\\
$n2=(\cancel{\frac{0}{(-1-0)(-1-1)}}+\frac{-1}{(0-(-1))(0-1)}+\cancel{\frac{0}{(1-(-1))(1-0)}})(x-(-1))(x-0)$\\
$n3=(\cancel{\frac{0}{(-1-0)(-1-1)(-1-2)}}+\frac{-1}{(0-(-1))(0-1)(0-2)}+\cancel{\frac{0}{(1-(-1))(1-0)(1-2)}}+\frac{1}{(2-(-1))(2-0)(2-1)})(x-(-1))(x-0)(x-1)$\\
$P(x)=n0+n1+n2+n3$\\
$P(x)=\frac{1}{3}(-x^3+3x^2+x-3)$\\
Ogólnie Newton to suma wielomianów k-tego stopnia, gdzie nk to suma $\frac{f(x_k)}{x_k-x_n}$ pomnożona przez $(x-x_n)$, gdzie $n \in \{1,2...,k\}$ (zerowy wielomian to $f(x_0)$).

\section{Metody rozwiązywania układów}
\subsection{Metoda Newtona (Nieliniowe)}
Wzór: $N(x)=x-\frac{f(x)}{f'(x)}$\\
Uwaga: chyba można znaleźć punkt $g(x)=0$ z $g(x)=\frac{f(x)}{f'(x)}$, wtedy tam jest albo $f(x)=0$ albo $f(x)=\pm \infty$, przynajmniej tak działa dla poniższych przykładów, ale wtedy to nie jest metoda Newtona chyba.
\subsubsection{$f(x)=ln(x)-\frac{1}{x}$}
$\frac{d}{dx}f(x)=f'(x)=\frac{x-1}{x^2}$\\
$x_0=2$\\
$x_1=N(x_0)=1.74247042592$\\
$x_2=N(x_1)=1.76305587394$\\
$x_3=N(x_2)=1.76322282359$\\
$x_4=N(x_3)=1.76322283435$\\
Nie chce mi się dalej robić, ale punkt zerowy jest w około $x=1.7632228$, co daje $f(1.7632228)=-3*10^-8$\\

\subsubsection{$f(x)=e^x+\frac{1}{x}$}
$\frac{d}{dx}f(x)=f'(x)=e^x-\frac{1}{x^2}$\\
$x_0=1$
$x_1=N(x_0)=-1.16395341374$\\
$x_2=N(x_1)=-2.44811693124$\\
$x_3=N(x_2)=-6.45348240853$\\
$x_4=N(x_3)=-13.2898033355$\\

Dalej nie robię, bo $x_n-x_{n-1}$ się zwiększa a nie zmniejsza, czyli funkcja nie przecina punktu 0.

\subsection{$f(x)=e^x-\frac{1}{x}$}
$\frac{d}{dx}f(x)=f'(x)=e^x+\frac{1}{x^2}$\\
$x_0=1$
$x_1=N(x_0)=0.53788284274$\\
$x_2=N(x_1)=0.566277007666$\\
$x_3=N(x_2)=0.567142580362$\\
$x_4=N(x_3)=0.567143290409$\\

Punkt zerowy znajduje się mniej więcej w x=0.56714

\subsection{Jacobiego (liniowe)}
Wzór: $x_{i+1}=(1-D^{-1}A)x_i+D^{-1}b$\\
Uwaga: Ta metoda działa tylko jeśli w każdym wierszu i kolumnie suma modułów elementów niediagonalnych jest mniejsza niż moduł elementu diagonalnego.
\subsubsection{\begin{equation*}\begin{cases}
    4x_1-2x_2=1\\
    -8x_1+x_2=3
\end{cases}\end{equation*}}
Dla tych danych nie możemy zastosować metody Jacobiego, ponieważ $|4|>|-2|$, ale $|1|\cancel{>}|-8|$. Teoretycznie możemy dodać do 2giego równania 2*pierwsze, wtedy warunek będzie spełniony:

\begin{equation*}\begin{cases}
    4x_1-2x_2=1\\
    0x_1-3x_2=5
\end{cases}\end{equation*}

Z czego wynika że:
\begin{equation*}\begin{cases}
    x_1=-\frac{7}{12}\\
    x_2=-\frac{5}{3}
\end{cases}\end{equation*}

Ale udajemy że jesteśmy komputerami i tego nie wiemy.

Zapisujemy równanie do równania z macierzami:\\
$Ax=b\:=>\:\begin{bmatrix} 4 & -2 \\ 0 & -3 \end{bmatrix} \begin{bmatrix} x_1 \\ x_2 \end{bmatrix} = \begin{bmatrix} 1 \\ 5 \end{bmatrix}$

Jako wektor startowy przyjmiemy:\\
$x_0=\begin{bmatrix} 0 \\ 0 \end{bmatrix}$

Tworzymy macierz $D^{-1}$ (Odwrotności diagonalnych macierzy powyżej):\\
$D^{-1}=\begin{bmatrix} 4 & 0 \\ 0 & -3 \end{bmatrix}^{-1} = \begin{bmatrix} \frac{1}{4} & 0 \\ 0 & -\frac{1}{3} \end{bmatrix}$

Wyliczymy od razu $D^{-1}b$ oraz $1-D^{-1}A$, ponieważ pozostają niezmienne pomiędzy iteracjami.

$D^{-1}b=\begin{bmatrix} \frac{1}{4} & 0 \\ 0 & -\frac{1}{3} \end{bmatrix}*\begin{bmatrix} 1 \\ 5 \end{bmatrix} = \begin{bmatrix} \frac{1}{4} \\ -\frac{5}{3} \end{bmatrix}$\\
$1-D^{-1}A = \begin{bmatrix} 1 & 0 \\ 0 & 1 \end{bmatrix}-\begin{bmatrix} \frac{1}{4} & 0 \\ 0 & -\frac{1}{3} \end{bmatrix}*\begin{bmatrix} 4 & -2 \\ 0 & -3 \end{bmatrix} = \begin{bmatrix} 0 & \frac{1}{2} \\ 0 & 0 \end{bmatrix}$\\

Kuaaro fact: Pamiętajcie kochani, że jak w Wolframie zrobicie $1-A$, to to nie jest $\begin{bmatrix} 1 & 0 \\ 0 & 1 \end{bmatrix} - A$ tylko $\begin{bmatrix} 1 & 1 \\ 1 & 1 \end{bmatrix} - A$ (Cries in a lost hour).

Wzór iteracyjny dla tych równań: $x_{i+1}=\begin{bmatrix} 0 & \frac{3}{2} \\ 1 & 0 \end{bmatrix}*x_i+\begin{bmatrix} \frac{1}{4} \\ -\frac{5}{3} \end{bmatrix}$\\
$x_1=\begin{bmatrix} 0 & \frac{1}{2} \\ 0 & 0 \end{bmatrix}*\begin{bmatrix} 0 \\ 0 \end{bmatrix}+\begin{bmatrix} \frac{1}{4} \\ -\frac{5}{3} \end{bmatrix} = \begin{bmatrix} \frac{1}{4} \\ -\frac{5}{3} \end{bmatrix}$\\
$x_2=\begin{bmatrix} 0 & \frac{1}{2} \\ 0 & 0 \end{bmatrix}*\begin{bmatrix} \frac{1}{4} \\ -\frac{5}{3} \end{bmatrix}+\begin{bmatrix} \frac{1}{4} \\ -\frac{5}{3} \end{bmatrix} = \begin{bmatrix} -\frac{7}{12} \\ -\frac{5}{3} \end{bmatrix}$\\
$x_3=\begin{bmatrix} 0 & \frac{1}{2} \\ 0 & 0 \end{bmatrix}*\begin{bmatrix} -\frac{7}{12} \\ -\frac{5}{3} \end{bmatrix}+\begin{bmatrix} \frac{1}{4} \\ -\frac{5}{3} \end{bmatrix} = \begin{bmatrix} -\frac{7}{12} \\ -\frac{5}{3} \end{bmatrix}$\\

Skoro iteracje się nie zmieniają, mamy rozwiązanie:

\begin{equation*}\begin{cases}
    x_1=-\frac{7}{12}\\
    x_2=-\frac{5}{3}
\end{cases}\end{equation*}

\subsection{Dekompozycja LU (liniowe)}
Ogólny wzór: $Ax=b\:=>\:LRx=b$, gdzie $LR=A$, L i R to macierze odpowiednio dolno i górno trujkątne.\\
Rozwiązać:

\begin{equation*}\begin{cases}
    x_1-x_2-x_3=1\\
    -x_1+x_2-x_3=2\\
    -x_1+x_2+x_3=-1
\end{cases}\end{equation*}

\subsubsection{Metoda Crouta-Doolittle}
Będę rozwiązywać wg. książki, bo mi się nie chce rozumieć, co się dzieje.
$ $

\end{document}